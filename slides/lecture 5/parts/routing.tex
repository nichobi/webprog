%--- routing ------------------------------------------------------------------------------
\section{Routing}
%--- routing------------------------------------------------------------------------------
\begin{frame}[fragile] \frametitle{Routing}
\begin{itemize}
  \item the browser history is part of the user experience
  \item allows the user to navigate back to earlier visited pages
  \item an entry in the history is added when the user
  \begin{itemize}
    \item navigates to a new page using a link
    \item submits a form
  \end{itemize}
  \item traditionally, this loads a new page from the server
  \item when a new page is loaded, all JavaScript objects are lost
  \item singel page web application prevents this using \code{preventDefault()} on all relevant events
  \item only updating the DOM will impact the user experience:
  \begin{itemize}
    \item can not navigate using the browser history (back button)
    \item can not link to inner pages
  \end{itemize}
\end{itemize}
\end{frame}

%--- Routing Framework------------------------------------------------------------------------------
\begin{frame}[fragile] \frametitle{Routing Framework}
\begin{itemize}
  \item there is an API giving JavaScript direct access to the browser history
  \item using it manually is tedious and error prone
  \item let a router do the work for you
  \item a router have two main features: link and route
\end{itemize}

\vspace{10mm}
\code{npm install react-router-dom}
\end{frame}

%--- Link ------------------------------------------------------------------------------
\section{Link}
%--- Link ------------------------------------------------------------------------------
\begin{frame}[fragile] \frametitle{Link}
\code{<Link to="/animals">animals</Link>}
\\ \code{<Link to="animals/fish">fishs</Link>}
\begin{itemize}
    \item let users navigate in your app
    \item clicking on it will add an entry to the browser history
    \item this will update the url field in the browser
    \item no new page is fetched from the server
    \item your JavaScript objects are untouched (preserve the application state)
\end{itemize}
\end{frame}

%--- NavLink ------------------------------------------------------------------------------
\begin{frame}[fragile] \frametitle{NavLink}
Add styling of active link using:
\begin{itemize}
  \item adda a css class to the element
  \item \code{<NavLink to="animals" />Animals</Link>}
  \item \code{<NavLink to="animals/fish" className=\{styleFunction\} />}
\end{itemize}

\begin{CodeBox}{}
function styleFunction(isActive) {
  if(isActive) {
    return "active";
  } else {
    return undefined;
  }
}
\end{CodeBox}
\end{frame}

%--- Link Example------------------------------------------------------------------------------
\begin{frame}[fragile] \frametitle{Link Example}
\begin{CodeBox}{}
import { Link } from 'react-router-dom';
function Menu() {
  return (
    <nav>
      <Link to="animal" />
      <Link to="animal/fish" />
      <Link to="animal/bird" />
    </nav>
  );
}
\end{CodeBox}
\end{frame}

%--- Route ------------------------------------------------------------------------------
\section{Route}
%--- Route ------------------------------------------------------------------------------
\begin{frame}[fragile] \frametitle{Route}
\code{<Route path="/" component=\{<h1>Hello</h1>\} />}
\begin{itemize}
  \item renders components based on url matching
  \item renders the \code{component} if \code{path} matches the current browser url
  \item wildcard: * matches any url
  \item \code{<Route>} must be direct child of \code{<Routes>} or \code{<Route>}
  \item the user can bookmark or copy the url to any page in your application
\end{itemize}
\end{frame}

%--- Nested Routs ------------------------------------------------------------------------------
\begin{frame}[fragile] \frametitle{Nested Routs}
Routs can be nested:
\begin{itemize}
  \item each level matches a part of the path
  \item at most one sibling \code{Route} will be matched
  \item all matched \code{Route element}s will be rendered
  \item most specific match wins
  \item wildcard have lowest priority
  \item index route:\\ \code{<Route index element=\{<SelectAnimal />\} />}
  \begin{itemize}
    \item selected if parent is matched but no siblings
    \item do not have any \code{path}
    \item do have a \code{index} attribute
  \end{itemize}
\end{itemize}
\end{frame}


%--- Route example------------------------------------------------------------------------------
\begin{frame}[fragile] \frametitle{Route Example}
\vspace{-3mm}
\begin{CodeBox}{}
import { Route, Routes } from 'react-router-dom';
function App() {
  return (
    <Routes>
      <Route path="animal" element={<Animal />}>
        <Route path="fish"} element={<Fish />}/>
        <Route path="bird"} element={<Bird />}/>
        <Route index element={<SelectAnimal />}/>
      </Route>
    </Routes>
  );
}
\end{CodeBox}
\vspace{-5mm}
/animal/fish $\rightarrow$ \code{<Animal><Fish /><Animal>}
\\/animal/cat $\rightarrow$ no match
\end{frame}

%--- Outlet ------------------------------------------------------------------------------
\begin{frame}[fragile] \frametitle{\code{<Outlet />}}
The child element is rendered in the parents \code{<Outlet />}
\vspace{5mm}
\begin{CodeBox}{}
import { Outlet } from 'react-router-dom';
function Animal() {
  return (
    <div>
      <h1>Animals</h1>
      <Outlet />
    </div>
  );
}
\end{CodeBox}
\end{frame}

%--- Path Parameters ------------------------------------------------------------------------------
\begin{frame}[fragile] \frametitle{Path Parameters}
the router pass data from the path to the component
\begin{itemize}
  \item specify parameters in the \code{path} using the syntax\code{:name}
  \item use the \code{useParams()}  hook to get an object with the values
\end{itemize}

\vspace{5mm}
\begin{CodeBox}{}
import { useParams } from "react-router-dom";

const elem = 
  <Route path="/item/:itemId" element={<Item />} />;

function Item() {
  let params = useParams();
  return <h2>item: {params.itemId}</h2>;
}
\end{CodeBox}
\end{frame}

%--- Route example again------------------------------------------------------------------------------
\begin{frame}[fragile] \frametitle{Route Example again}
\begin{CodeBox}{}
import { Route, Routes } from 'react-router-dom';
function App() {
  return (
    <Routes>
      <Route path="animal" element={<Animal />}>
        <Route path=":animal"} element={<ShowAnimal />}/>
        <Route index element={<SelectAnimal />}/>
      </Route>
    </Routes>
  );
}
\end{CodeBox}
/animal/fish $\rightarrow$ \code{<Animal><ShowAnimal /><Animal>}
\\/animal/cat $\rightarrow$ \code{<Animal><ShowAnimal /><Animal>}
\end{frame}

%--- Hooks ------------------------------------------------------------------------------
\begin{frame}[fragile] \frametitle{Hooks}
\begin{itemize}
  \item can be used in any child of \code{Route}
  \item \code{useParams()} - returns an object with the URL path parameters
  \item \code{useLocation()} - returns the browser location
  \item \code{useSearchParams()} - interact with query string in the URL
  \item \code{useNavigate()} - navigate programatically
\end{itemize}
\vspace{5mm}
Limitation: can only be used in functional components
\end{frame}

%--- Router ------------------------------------------------------------------------------
\begin{frame}[fragile] \frametitle{Router}

\begin{itemize}
  \item any web application starts with a html file \code{index.html} 
  \item the html file includes the react JavaScript code
  \item to allow direct links: the server must return \code{index.html} for all URLs
\end{itemize}

\code{<BrowserRouter>}
\begin{itemize}
  \item \code{http://domain.se/item/42}
  \item node.js built in server do this for you
  \item apache can be configured with redirect
\end{itemize}

\code{<HashRouter>}
\begin{itemize}
  \item \code{http://domain.se/#/item/42}
  \item compatible with all servers
\end{itemize}
\end{frame}

