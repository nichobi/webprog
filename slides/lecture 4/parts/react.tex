%--- react------------------------------------------------------------------------------
\begin{frame}[fragile] \frametitle{React}
\begin{CodeBox}{}
ReactDOM.render(
  <h1>Hello, world!</h1>,
  document.getElementById('root')
);
\end{CodeBox}

\vspace{8mm}
\begin{itemize}
  \item templates are JSX expressions
  \item JSX is embedded in the JavaScript code
  \item babel translates JSX to JavaScript code - builds a tree of React objects
  \item React injects the template object tree into the DOM
  \item rebuilt when a change happens
\end{itemize}

\end{frame}

%--- JSX ------------------------------------------------------------------------------
\section{JSX}
%--- JSX------------------------------------------------------------------------------
\begin{frame}[fragile] \frametitle{JSX}
\begin{itemize}
  \item any ``html tag'' is a JSX expression
  \item all tags must be closed, xml syntax works: \code{<img />}
  \item tags can be nested
  \item must be one root tag
  \item can use multiple lines
  \item use () to avoid automatic semicolon insertion!!!
  \item Babel compiles JSX down to \code{React.createElement()} calls
\end{itemize}

\begin{CodeBox}{}
const element = (
<span>
  <h1>Hello, world!</h1>
  <p>Some more text...</p>
</span>);
\end{CodeBox}
\end{frame}

%--- JSX attributes------------------------------------------------------------------------------
\begin{frame}[fragile] \frametitle{JSX attributes}
\begin{itemize}
  \item JSX tags can have attributes
  \item React DOM uses camelCase
  \item html: \code{class}, JSX: \code{className}
  \item html: \code{for}, JSX: \code{htmlFor}
  \item mapps to html attributes when injected to the DOM
\end{itemize}

\vspace{8mm}
\begin{CodeBox}{}
const element1 = <div tabIndex="0"></div>;
const element2 = <img src="picture.jpeg" />;
\end{CodeBox}
\end{frame}

%--- Embedded JavaScript ------------------------------------------------------------------------------
\begin{frame}[fragile] \frametitle{Embedded JavaScript}
JSX can contain embedded JavaScript
\begin{itemize}
  \item syntax: \{ \emph{JavaScript expression} \}
  \item use in:
  \begin{itemize}
    \item attribute values
    \item tag content
  \end{itemize}
  \item the embedded JavaScript expression may evaluate to another JSX expression
\end{itemize}
\vspace{3mm}
\begin{CodeBox}{}
const name = 'Per';
const imgSrc = 'picture.jpeg';
function doSomeMagic(arg) { return <span>a frog</span> };
const element1 = <h1>Hello {name}</h1>;
const element2 = <img src={imgSrc} />;
const element3 = (<span>{element1}
  <p>You are transformed to {doSomeMagic(name)}</p></span>);
\end{CodeBox}
\end{frame}

%--- React Components ------------------------------------------------------------------------------
\section{React Components}
%--- React Components------------------------------------------------------------------------------
\begin{frame}[fragile] \frametitle{Rect Components}
react components
\begin{itemize}
  \item gives a name to a template
  \item the name can be used as a tag in JSX
  \item must start with a capital letter
  \item is a JavaScript object:
  \begin{itemize}
    \item a template: a function returning an JSX expression
    \item applications state
    \item mechanism for component-component communcation
  \end{itemize}
  \item Model-View design pattern
\end{itemize}
\end{frame}

%--- Component example------------------------------------------------------------------------------
\begin{frame}[fragile] \frametitle{Component Example}
\begin{CodeBox}{}
  function HelloWorld(){
    return (
      <span>
        <h1>Hello</h1>
        <p>EDAF90 - web programming</p>
      </span>
    );
  }
\end{CodeBox}
\end{frame}
%--- Attributes------------------------------------------------------------------------------
\begin{frame}[fragile] \frametitle{Attributes - \code{props}}
a parent can pass data to a child
\begin{itemize}
  \item parent set the value using attributes in JSX
  \item the values are passed as a \code{props}
\end{itemize}
\end{frame}

%--- Properties Example ------------------------------------------------------------------------------
\begin{frame}[fragile] \frametitle{Properties Example}

\begin{CodeBox}{}
function HelloWorld(props) {
  return (
    <span>
      <h1>Hello World</h1>
      <p>{props.message}</p>
    </span>
  );
}

const element = (<HelloWorld 
                   message="EDAF90 - web programming"/>);
\end{CodeBox}
\end{frame}

%--- Child Parent Communication ------------------------------------------------------------------------------
\begin{frame}[fragile] \frametitle{Child Parent Communication}
\begin{itemize}
  \item properties pass data down in the tree
  \item can also be used to pass data up the tree
  \begin{itemize}
    \item parent have a call back function to update its state
    \item pass it as a property a child
    \item the child calls the parent call back function
  \end{itemize}
\end{itemize}
\end{frame}


%--- State ------------------------------------------------------------------------------
\section{State}
%--- React Components------------------------------------------------------------------------------
\begin{frame}[fragile] \frametitle{Rect Components}
stateless component
\begin{itemize}
  \item a function returning a react element (JSX expression)
  \item should be pure functions
  \item or, use hooks to manage state
\end{itemize}

stateful component
\begin{itemize}
  \item is a ES6 class with these properties:
  \begin{itemize}
    \item \code{extends React.Component}
    \item \code{constructor(props)} initialise the state
    \item \code{render()} returns a react element (JSX expression)
    \item \code{this.state} - an object containing the component state
    \item \code{this.state} - must be an immutable object
  \end{itemize}
\end{itemize}
\end{frame}

%--- State Example ------------------------------------------------------------------------------
\begin{frame}[fragile] \frametitle{State Example}
\begin{CodeBox}{}
class Hello extends React.Component {
  constructor(props) {
    super(props);
    this.state = {name: props.name + ' Andersson'};
  }
  render() {
    return (
      <span>
        <h1>Hello {this.state.name}</h1>
        <p>{this.props.message}</p>
      </span>
  }
}
const element = <Hello name="Per" />;
\end{CodeBox}
\end{frame}

%--- State------------------------------------------------------------------------------
\begin{frame}[fragile] \frametitle{State}
\begin{itemize}
  \item the state object is immutable
  \item use \code{setState()} to update state
  \item state is updated asynchronously (at some time in the future)
  \item use functions for incremental update: \code{(state, props) => (\{...state, \})}
  \item state updates are merged
  \begin{itemize}
    \item the new state contains the state properties you want to change
    \item all other state properties remain unchanged
    \item you can not remove properties
    \item set undesired properties to \code{undefined}
  \end{itemize}
\end{itemize}
\end{frame}

%--- Render Elements------------------------------------------------------------------------------
\begin{frame}[fragile] \frametitle{Render Elements}
\code{ReactDOM.render()}
\begin{itemize}
  \item updates the DOM
  \item takes two parameters:
  \begin{itemize}
    \item a react element (JSX expression)
    \item a DOM element
  \end{itemize}
  \item updates the DOM if the element already was part of the DOM
  \item optimised, only updated the delta
  \item 
\end{itemize}
\end{frame}

%--- Render Elements------------------------------------------------------------------------------
\begin{frame}[fragile] \frametitle{Render Elements}
\begin{itemize}
\item \code{state} - an immutable object
\item never change the state object, or any object it refers to!
\item \code{setState()} - change the state of a component
\begin{itemize}
  \item \code{asyncrounous}
  \item will updates the component eventually
  \item will updates the DOM
  \item optimised, only updated the delta
\end{itemize}
\end{itemize}
\end{frame}

%--- Event Handling ------------------------------------------------------------------------------
\section{Event Handling}
%--- Handling Events------------------------------------------------------------------------------
\begin{frame}[fragile] \frametitle{Handling Events}
\begin{itemize}
  \item names are camelCase
  \item must use \code{preventDefault}, returning \code{false} do nothing
  \item pass a function: \code{onClick =\{myCallbackFunction\}}
  \item argument is a synthetic event according to the W3C spec
\end{itemize}
\begin{CodeBox}{}
function ActionLink() {
  function handleClick(e) {
    e.preventDefault();
    console.log('The link was clicked.');
  }
  return (
    <a href="#" onClick={handleClick}>Click me</a>
  );
}
\end{CodeBox}
\end{frame}

%--- About this------------------------------------------------------------------------------
\begin{frame}[fragile] \frametitle{About this}
\begin{itemize}
  \item by default \code{this} is \code{undefined} in a callback function
  \item use \code{bind()} to bind it to the component object
  \item or arrow functions
\end{itemize}
\begin{CodeBox}{}
inventory.map(row => { return (
    <button onClick={(event) => this.deleteRow(row, event)}>
      Delete {row.name}
    </button>
)};
\end{CodeBox}
\end{frame}
%--- this example------------------------------------------------------------------------------
\begin{frame}[fragile] \frametitle{this example}
\begin{CodeBox}{}
class Counter extends React.Component {
  constructor(props) {
    super(props);
    this.state = {count: 0};
    this.handleClick = this.handleClick.bind(this);
  }
  handleClick() {
    this.setState(state => ({count: state.count + 1});
  }
  render() {
    return (
      <button onClick={this.handleClick}>
        {this.state.count}
      </button>
    );
  }
}
\end{CodeBox}
\end{frame}

%--- Lists and key ------------------------------------------------------------------------------
\section{Lists and key}
%--- Lists and key------------------------------------------------------------------------------
\begin{frame}[fragile] \frametitle{Lists and key}
\begin{itemize}
  \item embedded JavaScript may return a collection of react elements
  \item added in sequence to the DOM
  \item to help the render optimisation:
  \begin{itemize}
    \item each element should have a \code{key} property
    \item unique among siblings
    \item the value must be preserved over time
    \item avoid array index as key (changes when elements are deleted)
  \end{itemize}
\end{itemize}
\end{frame}

%--- Key example------------------------------------------------------------------------------
\begin{frame}[fragile] \frametitle{\code{key} example}

\begin{CodeBox}{}
function MyList(props) { return (
  <ul>
    props.list.map(row => (
      <li key={row.id}>
        {row.text}
      </li>
    ))
  </ul>);
}
\end{CodeBox}
\end{frame}

%--- Forms------------------------------------------------------------------------------
\begin{frame}[fragile] \frametitle{Forms}
\begin{itemize}
  \item DOM forms contain state
  \item hard for you JavaScript code to read that state
  \item use the \emph{controlled component} pattern
  \begin{itemize}
    \item the react component have the master state
    \item the DOM reflects the component state using JSX embedded JavaScript
    \item catch events (\code{onChange}): update the component state
  \end{itemize}
  \item \code{<form onSubmit=\{this.handleSubmit\}>} \ldots \code{</form>}
\end{itemize}
\vspace{5mm}
\begin{CodeBox}{}
<input type="text" 
        value={this.state.value}
        onChange={this.handleChange}
/>
\end{CodeBox}
\end{frame}

%--- Forms, select------------------------------------------------------------------------------
\begin{frame}[fragile] \frametitle{Form, select}
\begin{itemize}
  \item html, uses \code{selected} attribute in an \code{option} tag
  \item react uses a \code{value} attribute in the \code{select} tag
\end{itemize}
\vspace{3mm}
html
\begin{CodeBox}{}
<select>
  <option value="Coconut">Coconut</option>
  <option selected value="lime">Lime</option>
</select>\end{CodeBox}
\vspace{3mm}
react
\begin{CodeBox}{}
<select value={this.state.value} onChange={this.handler}>
  <option value="lime">Lime</option>
  <option value="coconut">Coconut</option>
</select>\end{CodeBox}
\end{frame}

%--- Forms, textarea------------------------------------------------------------------------------
\begin{frame}[fragile] \frametitle{Form, textarea}
\begin{itemize}
  \item html, value is the content
  \item react uses a \code{value} property
\end{itemize}
\vspace{3mm}
html
\begin{CodeBox}{}
<textarea>
  Hello there, some text in a text area
</textarea>
\end{CodeBox}
\vspace{3mm}
react
\begin{CodeBox}{}
const text = 'Hello there, some text in a text area';
const elem = (
  <textarea value={this.state.value}
               onChange={this.handleChange} />;
\end{CodeBox}
\end{frame}

%--- Forms, file input------------------------------------------------------------------------------
\begin{frame}[fragile] \frametitle{Form, file input}
\begin{CodeBox}{}
<input type="file"/>
\end{CodeBox}
\vspace{8mm}

\begin{itemize}
  \item is read-only
  \item must be uncontrolled
\end{itemize}
\end{frame}

%--- Forms call back------------------------------------------------------------------------------
\begin{frame}[fragile] \frametitle{Form, call back}
\begin{CodeBox}{}
class NameForm extends React.Component {
  constructor(props) {
    super(props);
    this.state = {value: ''};
    this.handleChange = this.handleChange.bind(this);
    this.handleSubmit = this.handleSubmit.bind(this);
  }

  handleChange(event) {
    this.setState({value: event.target.value});
  }
  handleSubmit(event) {
    alert('A name was submitted: ' + this.state.value);
    event.preventDefault();
  }
\end{CodeBox}
\end{frame}

%--- Forms, generic callback------------------------------------------------------------------------------
\begin{frame}[fragile] \frametitle{Forms, generic callback}
\begin{CodeBox}{}
handleInputChange(event) {
  const target = event.target;
  const value =
    target.type === 'checkbox' ?
                    target.checked : target.value;
  const name = target.name;

  this.setState({
    [name]: value
  });
}
\end{CodeBox}
\end{frame}

%--- Form refs------------------------------------------------------------------------------
%\begin{frame}[fragile] \frametitle{Form refs}
%\begin{CodeBox}{}
%const input = React.createRef();
%function handleSubmit(event) {
%  alert('A name was submitted: ' + input.current.value);
%  event.preventDefault();
%}
%
%const form = (
%  <form onSubmit={this.handleSubmit}>
%    <label> Name:
%      <input type="text" ref={input} />
%    </label>
%    <input type="submit" value="Submit" />
%  </form>);
%\end{CodeBox}
%\end{frame}

%--- Life Cycle Hooks------------------------------------------------------------------------------
\begin{frame}[fragile] \frametitle{Life Cycle Hooks}
\begin{CodeBox}{}
class Hello extends React.Component {
  constructor(props) {
    super(props);
  }

  componentDidMount() {  }

  componentWillUnmount() {  }

  render() {
    return <h1>Hello, world!</h1>;
  }
}
\end{CodeBox}
\end{frame}
